
% Define state styles
\tikzstyle{resource} = [circle, draw, top color=white, bottom color=red!20, draw=red!50!black!100, drop shadow, text width=2cm, text centered, node distance=2cm, minimum height=2cm]
\tikzstyle{bee} = [rectangle, draw, top color=white, bottom color=blue!30, draw=blue!50!black!100, drop shadow, text width=2cm, text centered, rounded corners, minimum height=2cm]
\tikzstyle{environment} = [circle, draw, top color=white, bottom color = green!30, draw=green!50!black!100, drop shadow, text width=2cm, text centered, rounded corners, minimum height=2cm]
\tikzstyle{line} = [line width=1.5pt, draw, -latex']


\begin{figure}
	\centering
	\scalebox{0.9}{
	\small
	\begin{tikzpicture}[node distance = 3cm and 3cm, auto]
		% Place nodes
		\node [bee] (queen) {Queen};
		\node [bee, below of=queen, node distance = 4cm] (brood) {Uncapped brood};
		\node [bee, below of=brood, node distance = 4cm] (hiveBees) {Hive bees};
		\node [bee, below of=hiveBees, node distance = 4cm] (foragers) {Foragers};
		%\node []
		%\node [resource] (unemployed) {Unemployed bee};
		%\node [bee, right of=unemployed, node distance = 3.5cm] (moreBees) {More active bees possible?};
		%\node [environment, right of=moreBees, node distance = 4cm] (moreScouts) {More scouts possible?};
		% Draw edges
		%\path [line] +(-2cm, 0.0) -- (unemployed.north);
%		\path [line] (queen.south) -- node {eclosion} (brood.north);
		\path [line] (brood.south) -- node {} (hiveBees.north);
		\path [line] (hiveBees.south) edge [bend left] node {food income requirements} (foragers.north);
		\path [line] (foragers.north) edge [bend left] node {social inhibition} (hiveBees.south);
	\end{tikzpicture}}
	\caption{\textit{Advanced honey bee social dynamics covered by this model}}
	\label{fig:modelOverview}
\end{figure}

