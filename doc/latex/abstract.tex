\section*{Abstract}
In this article, we present a model that simulates the behaviour of a single hive of honey bees (\textit{Apis mellifera}), interacting with its environment. Three different kinds of bees were implemented: The forager, the scout bee and the hive bee. The hive is a quantitative model substantially based on the works of \textit{David S. Khoury et al.} \cite{khoury13} using a range of differential equations to represent hive growth, social inhibition, food availability and bee ageing. The focus of this model lies, in contrast to \textit{Khoury et al.}, in the hive's interaction with the environment and it's dependency on the seasonal changes (no food income in winter, different flowers each month). To simulate seasons, a fine-grained and easily extensible environment algorithm is used, in which the bees navigate and collect food. For the development of the environment simulation, empirical observations by \textit{T.D. Seeley} \cite{seeley95} were used.