\section{Abstract}
The global decrease of honey bee populations (\textit{Apis mellifera}) is alarming. Our dependency on pollination in different kinds of agricultural areas (such as fruit, vegetables, spices, nuts etc.) urges us to find out more about the bee's behaviour and their mortality. In recent years, many studies $^{[citation.needed]}$ have been carried out to find out the reasons for the mass decline of pollinators: diseases, destruction of the environment, pesticides/fungicides (imidacloprid, thiamethoxam, clothianidin, fipronil, chlorpyriphos, cypermethrin and deltamethrin) and maybe even climate change. The sum of all these factors culminates in the mass death of bees and can lead to a serious economic crash ().\\
In our model, we simulate the behaviour of a single hive, interacting with it's environment. Three different kinds of bees were implemented: The forager, the scout bee and the hive bee. \\
$[...]$\\
The focus of the model lies in the hive's interaction with the environment and it's dependency on the seasonal changes (no food income in winter, different flowers in each month). The plants have been implemented in respect of the diffusion rate of the scent and as a function of quality and season. Additionally, it'd be possible to realize a CO$_{2}$ gradient, to generate the data even more precisely. Our hive is a quantitative model based on the works of David S. Khoury et al. $^[citation needed]$ using a range of differential equations to represent hive growth, social inhibition, food availability and bee ageing.  
\\
$[...]$\\


