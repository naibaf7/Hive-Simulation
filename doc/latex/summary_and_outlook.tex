\section{Summary and Outlook}
\subsection{Realistic flowers and maps}
	To increase the accuracy of our model further, the flower blooming functions could be split up more minutely. They could be replaced by more accurate quantitative relations, than those we used. Moreover, flower distribution was computer generated; it would be very interesting to create maps of real places (agricultural areas vs. urban and suburban areas, maps near heavily used roads etc.) and compare their results.
\subsection{Improvements to the hive model}
	The model would actually support multiple hives with different parameters. For example, various bee species could be simulated on the same map contesting for food. \\Furthermore, we never tried to implement swarming in our model. With the model's ability to handle multiple hives one could implement swarming and find out more about the evolution of bee hives and their long term behaviour. \\Some of our measures to reduce run-time (clustering, bigger time steps) could be reversed to increase the model's accuracy (runtime can go up to 10 hours for a simulated year with the most precise settings (as of 2013)).
\subsection{Summary}
	Overall we found that bees can compensate short term food problems quite well by sacrificing larvae and recruiting bees for the most important tasks. They can not compensate however, the loss of a whole season (e.g. summer or fall) or a substandard flower blooming in summer or fall.\\ Interestingly enough, most of our calculated and simulated data correlate nicely with the literature of our two basis papers (\textit{Khoury et al.} \cite{khoury13} and \textit{T.D. Seeley} \cite{seeley95}), even though we took values from many different sources.
\subsection{Outlook}
	In our rapidly developing world, studies like this become more and more important. Things like bees, seemingly 'unnecessary' insects, have a very big influence on our way of life - and vice versa. It is intriguing to see, how a simple model, compared to nature, can predict events so precisely and so fallaciously at the same time. Models are always extensible and improvable, which is exactly what should be done. Only with science, we can find out more about the world.  