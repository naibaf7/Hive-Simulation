\section{Summary and Outlook}
\subsection{Realistic flowers and maps}
	To further increase the accuracy of our model, the flower blooming functions could be split up more minutely and replaced by more accurate ones than those we used. Our map was a computer generated map with equally distributed flower patches. It would be an interesting idea to create maps of real places (agricultural, urban and near cities, near heavily used roads) and compare their results.
\subsection{Improvements to the hive model}
	The model would actually support multiple hives with different parameters. For example, different bee species could be simulated on the same map contesting for food. \\Furthermore we never tried to implement swarming in our model. But by implementing swarming, and the model's already existing capabilities to handle multiple hives, an evolutionary simulation over a long time span could lead to interesting results.\\Some of our measurements to reduce run-time (clustering, bigger time steps) could be reversed to increase the model's accuracy (runtime can go up to 10 hours for a simulated year with the most precise settings (as of 2013)).
\subsection{Summary}
	Overall we found that bees can compensate short term food problems quite well by sacrificing larvae and recruiting bees for the most important tasks. They can not compensate however, the loss of a whole season(eg. summer or fall) or a substandard flower blooming in summer or fall.