\section{Introduction and Motivations}
The global decrease of honey bee populations (\textit{Apis mellifera}) is not only conspicuous, but also alarming. Our dependency on pollination in almost all forms of agricultural cultivation (such as fruit and vegetables, spices, nuts etc.) urges us to find out more about the bee's behaviour and their mortality. In recent years, many studies\cite{potts10}\cite{thomann13} have been carried out to find out the reasons for the mass decline of pollinators: diseases, destruction of the environment, pesticides/fungicides (imidacloprid, deltamethrin \cite{decourtye04}, thiamethoxam, clothianidin, fipronil, chlorpyriphos, cypermethrin and deltamethrin) and maybe even climate change $^{[citation.needed]}$. The sum of all these factors culminates in the mass death of bees and can lead to a serious economic crash as well as to a impairment of agricultural resources.\\
In our model, we simulate the behaviour of a single hive, interacting with its environment. Three different kinds of bees were implemented: The forager, the scout bee and the hive bee. The focus of the model lies in the hive's interaction with the environment and it's dependency on the seasonal changes (no food income in winter, different flowers in each month). Our hive is a quantitative model mostly based on the works of David S. Khoury et al. \cite{khoury13} using a range of differential equations to represent hive growth, social inhibition, food availability and bee ageing. 