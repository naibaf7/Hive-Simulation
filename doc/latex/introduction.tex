\section{Introduction and Motivations}
The global decrease of honey bee populations (\textit{Apis mellifera}) is not only conspicuous, but also alarming. Our dependency on pollination in almost all forms of agricultural cultivation (such as fruit and vegetables, spices, nuts etc.) urges us to find out more about the bee's behaviour and their mortality. In recent years, many studies \cite{potts10} \cite{thomann13} have been carried out to find out the reasons for the mass decline of pollinators: diseases, destruction of the environment, pesticides/fungicides (such as imidacloprid, deltamethrin \cite{decourtye04}, fipronil \cite{bernadou09}, and many others) and maybe even climate change \cite{schweiger10}. The sum of all these factors culminates in the mass death of bees and can lead to a serious economic crash as well as to a impairment of agricultural resources.\\
This motivated us to develop a simulation to find corner situations in which the hive's health becomes unstable and does not survive. We started with an existing simulation by \textit{Khoury et al.} \cite{khoury13} and extended it with knowledge and empirical data from \textit{T.D. Seeley} \cite{seeley95}.